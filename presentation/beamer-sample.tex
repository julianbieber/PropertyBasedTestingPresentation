\documentclass[xcolor=svgnames,handout]{beamer}

\usepackage[utf8]    {inputenc}
\usepackage[T1]      {fontenc}
\usepackage[english] {babel}
\usepackage{graphicx}

\usepackage{amsmath,amsfonts,graphicx}
\usepackage{beamerleanprogress}


\title
[Fuzzy Testing \hspace{2em}]
{Fuzzy Testing \\ A golden bullet for writing tests?}

\author
[Julian Bieber]
{Julian Bieber}

\date
{\today}


\begin{document}

    \maketitle

    \section
    {Introduction}

    \begin{frame}
    {Definition}
        \begin{itemize}
            \item General input satisfying precondition
            \item Property that must hold true on output
        \end{itemize}
        \begin{figure}[ht]
        \centering
        \includegraphics[width=0.33\textwidth]{cat-10.png}
        \end{figure}

    \end{frame}


    \section
    {When to use Property Based Tests}

    \begin{frame}
    {Should every component be tested this way?}

        No, most small unit tests do not lend themselves well to Property Based Tests.

        In many cases the of writing Property Based Tests is hard to justify.

        But they excel in a few cases.
    \end{frame}


    \begin{frame}
    {Bijective Functions}
        \begin{itemize}
            \item Serialization - Serialization must be reversible
            \item DB read/write - Every object written to the db must be retrievable
            \item Generally applicable $f'(f(x)) = x$
        \end{itemize}

        Example: \texttt{bijective.SerializationSpec}
    \end{frame}


    \begin{frame}
    {Invariants}
        \begin{itemize}
            \item Algorithms - sorting, compression
            \item Repricing strategies
            \item Integration tests
            \item Generally applicable if there are simple Pre- and Postconditions
        \end{itemize}

        Example: \texttt{invariant.CompressionSpec}

    \end{frame}


    \begin{frame}
    {Regression}
        \begin{itemize}
            \item There is an old component$;$ no tests $:($
            \item Works mostly correct$;$ one known bug
            \item Goal: make sure new implementation does not break existing behavior
            \item $\forall x: notKnownBug(x) \implies old(x) = new(x)$
            \item $\forall x: previousBug(x) \implies expectation(new(x)) = true $
        \end{itemize}

        Example: \texttt{regression.FibonacciSpec}

    \end{frame}

    \begin{frame}
    {Notes On ScalaCheck - Preconditions}

        \begin{itemize}
            \item Bound to Generator with \texttt{Gen[A].suchThat(predicate)} or \texttt{Gen[A].retryUntil(predicate)}
            \item Generator built to include precondition \texttt{BasicGenerators.genNonEmptySeq[A](Gen[A]): Gen[Seq[A]]}
            \item Inside \texttt{forAll} block with \texttt{whenever(predicate(generated)) \{...\}}
        \end{itemize}

    \end{frame}

    \begin{frame}
    {Notes on ScalaCheck - Configuration}

        \begin{itemize}
            \item set through \texttt{implicit val PropertyCheckConfiguration}
            \item \texttt{minSuccessful}: number of successful evaluations for pass
            \item \texttt{maxDiscardedFactor}: \texttt{minSuccessful * maxDiscardFactor} = number of generated values that can be discarded
            \item \texttt{minSize}: minimun size for iterable types (String, Seq, Set ...)
            \item \texttt{sizeRange}: \texttt{minSize * sizeRange} = max size for iterable types
            \item \texttt{workers}: number of threads evaluating the \texttt{forAll} block
        \end{itemize}

    \end{frame}

    \begin{frame}
    {Notes on ScalaCheck - Shrink}

        \begin{itemize}
            \item Goal: Find smallest input to produce failure
            \item Bug: When a value is shrunk preconditions are ignored
            \item Bug: Input shrunk without failure
            \item Symptom: The test is executed for unexpected values $\implies$ test reporting failure where nothing is wrong
            \item Workaround: Disable shrinking when there are preconditions
            \item Workaround: Preconditions through specific generator + \texttt{whenever(p(x)) \{...\}} block
        \end{itemize}

        \texttt{ implicit val noShrink: Shrink[A] = Shrink.shrinkAny }

    \end{frame}

\end{document}

